\documentclass[avery5371,frame]{flashcards}
        \usepackage{graphicx,xcolor}
        \usepackage{mhchem}

        \cardfrontstyle{headings}

        \begin{document}\cardfrontfoot{Question 1}
        \begin{flashcard}[\tiny AMERICAN GOVERNMENT: A: Principles of American Democracy]{1. What is the supreme law of the land?}
        {the Constitution}
        \end{flashcard}\cardfrontfoot{Question 2}
        \begin{flashcard}[\tiny AMERICAN GOVERNMENT: A: Principles of American Democracy]{2. What does the Constitution do?}
        {sets up the government\\defines the government\\protects basic rights of Americans}
        \end{flashcard}\cardfrontfoot{Question 3}
        \begin{flashcard}[\tiny AMERICAN GOVERNMENT: A: Principles of American Democracy]{3. The idea of self-government is in the first three words of the Constitution. What are these words?}
        {We the People}
        \end{flashcard}\cardfrontfoot{Question 4}
        \begin{flashcard}[\tiny AMERICAN GOVERNMENT: A: Principles of American Democracy]{4. What is an amendment?}
        {a change (to the Constitution)\\an addition (to the Constitution)}
        \end{flashcard}\cardfrontfoot{Question 5}
        \begin{flashcard}[\tiny AMERICAN GOVERNMENT: A: Principles of American Democracy]{5. What do we call the first ten amendments to the Constitution?}
        {the Bill of Rights}
        \end{flashcard}\cardfrontfoot{Question 6}
        \begin{flashcard}[\tiny AMERICAN GOVERNMENT: A: Principles of American Democracy]{6. What is one right or freedom from the First Amendment?*}
        {speech\\religion\\assembly\\press\\petition the government}
        \end{flashcard}\cardfrontfoot{Question 7}
        \begin{flashcard}[\tiny AMERICAN GOVERNMENT: A: Principles of American Democracy]{7. How many amendments does the Constitution have?}
        {twenty-seven (27)}
        \end{flashcard}\cardfrontfoot{Question 8}
        \begin{flashcard}[\tiny AMERICAN GOVERNMENT: A: Principles of American Democracy]{8. What did the Declaration of Independence do?}
        {announced our independence (from Great Britain)\\declared our independence (from Great Britain)\\said that the United States is free (from Great Britain)}
        \end{flashcard}\cardfrontfoot{Question 9}
        \begin{flashcard}[\tiny AMERICAN GOVERNMENT: A: Principles of American Democracy]{9. What are two rights in the Declaration of Independence?}
        {life\\liberty\\pursuit of happiness}
        \end{flashcard}\cardfrontfoot{Question 10}
        \begin{flashcard}[\tiny AMERICAN GOVERNMENT: A: Principles of American Democracy]{10. What is freedom of religion?}
        {You can practice any religion, or not practice a religion.}
        \end{flashcard}\cardfrontfoot{Question 11}
        \begin{flashcard}[\tiny AMERICAN GOVERNMENT: A: Principles of American Democracy]{11. What is the economic system in the United States?*}
        {capitalist economy\\market economy}
        \end{flashcard}\cardfrontfoot{Question 12}
        \begin{flashcard}[\tiny AMERICAN GOVERNMENT: A: Principles of American Democracy]{12. What is the “rule of law”?}
        {Everyone must follow the law.\\Leaders must obey the law.\\Government must obey the law.\\No one is above the law.}
        \end{flashcard}\cardfrontfoot{Question 13}
        \begin{flashcard}[\tiny AMERICAN GOVERNMENT: B: System of Government]{13. Name one branch or part of the government.*}
        {Congress\\legislative\\President\\executive\\the courts\\judicial}
        \end{flashcard}\cardfrontfoot{Question 14}
        \begin{flashcard}[\tiny AMERICAN GOVERNMENT: B: System of Government]{14. What stops one branch of government from becoming too powerful?}
        {checks and balances\\separation of powers}
        \end{flashcard}\cardfrontfoot{Question 15}
        \begin{flashcard}[\tiny AMERICAN GOVERNMENT: B: System of Government]{15. Who is in charge of the executive branch?}
        {the President}
        \end{flashcard}\cardfrontfoot{Question 16}
        \begin{flashcard}[\tiny AMERICAN GOVERNMENT: B: System of Government]{16. Who makes federal laws?}
        {Congress\\Senate and House (of Representatives)\\(U.S. or national) legislature}
        \end{flashcard}\cardfrontfoot{Question 17}
        \begin{flashcard}[\tiny AMERICAN GOVERNMENT: B: System of Government]{17. What are the two parts of the U.S. Congress?*}
        {the Senate and House (of Representatives)}
        \end{flashcard}\cardfrontfoot{Question 18}
        \begin{flashcard}[\tiny AMERICAN GOVERNMENT: B: System of Government]{18. How many U.S. Senators are there?}
        {one hundred (100)}
        \end{flashcard}\cardfrontfoot{Question 19}
        \begin{flashcard}[\tiny AMERICAN GOVERNMENT: B: System of Government]{19. We elect a U.S. Senator for how many years?}
        {six (6)}
        \end{flashcard}\cardfrontfoot{Question 20}
        \begin{flashcard}[\tiny AMERICAN GOVERNMENT: B: System of Government]{20. Who is one of your state’s U.S. Senators now?*}
        {{\footnotesize{\textsl{Updated for VI on 2022-10-04}}}}
        \end{flashcard}\cardfrontfoot{Question 21}
        \begin{flashcard}[\tiny AMERICAN GOVERNMENT: B: System of Government]{21. The House of Representatives has how many voting members?}
        {four hundred thirty-five (435)}
        \end{flashcard}\cardfrontfoot{Question 22}
        \begin{flashcard}[\tiny AMERICAN GOVERNMENT: B: System of Government]{22. We elect a U.S. Representative for how many years?}
        {two (2)}
        \end{flashcard}\cardfrontfoot{Question 23}
        \begin{flashcard}[\tiny AMERICAN GOVERNMENT: B: System of Government]{23. Name your U.S. Representative.}
        {{\footnotesize{\textsl{Updated for VI on 2022-10-04}}}}
        \end{flashcard}\cardfrontfoot{Question 24}
        \begin{flashcard}[\tiny AMERICAN GOVERNMENT: B: System of Government]{24. Who does a U.S. Senator represent?}
        {all people of the state}
        \end{flashcard}\cardfrontfoot{Question 25}
        \begin{flashcard}[\tiny AMERICAN GOVERNMENT: B: System of Government]{25. Why do some states have more Representatives than other states?}
        {(because of) the state’s population\\(because) they have more people\\(because) some states have more people}
        \end{flashcard}\cardfrontfoot{Question 26}
        \begin{flashcard}[\tiny AMERICAN GOVERNMENT: B: System of Government]{26. We elect a President for how many years?}
        {four (4)}
        \end{flashcard}\cardfrontfoot{Question 27}
        \begin{flashcard}[\tiny AMERICAN GOVERNMENT: B: System of Government]{27. In what month do we vote for President?*}
        {November}
        \end{flashcard}\cardfrontfoot{Question 28}
        \begin{flashcard}[\tiny AMERICAN GOVERNMENT: B: System of Government]{28. What is the name of the President of the United States now?*}
        {Joseph R. Biden, Jr.\\Joe Biden\\Biden{\footnotesize{\textsl{Updated on 2022-10-04}}}}
        \end{flashcard}\cardfrontfoot{Question 29}
        \begin{flashcard}[\tiny AMERICAN GOVERNMENT: B: System of Government]{29. What is the name of the Vice President of the United States now?}
        {Kamala D. Harris\\Kamala Harris\\Harris{\footnotesize{\textsl{Updated on 2022-10-04}}}}
        \end{flashcard}\cardfrontfoot{Question 30}
        \begin{flashcard}[\tiny AMERICAN GOVERNMENT: B: System of Government]{30. If the President can no longer serve, who becomes President?}
        {the Vice President}
        \end{flashcard}\cardfrontfoot{Question 31}
        \begin{flashcard}[\tiny AMERICAN GOVERNMENT: B: System of Government]{31. If both the President and the Vice President can no longer serve, who becomes President?}
        {the Speaker of the House}
        \end{flashcard}\cardfrontfoot{Question 32}
        \begin{flashcard}[\tiny AMERICAN GOVERNMENT: B: System of Government]{32. Who is the Commander in Chief of the military?}
        {the President}
        \end{flashcard}\cardfrontfoot{Question 33}
        \begin{flashcard}[\tiny AMERICAN GOVERNMENT: B: System of Government]{33. Who signs bills to become laws?}
        {the President}
        \end{flashcard}\cardfrontfoot{Question 34}
        \begin{flashcard}[\tiny AMERICAN GOVERNMENT: B: System of Government]{34. Who vetoes bills?}
        {the President}
        \end{flashcard}\cardfrontfoot{Question 35}
        \begin{flashcard}[\tiny AMERICAN GOVERNMENT: B: System of Government]{35. What does the President’s Cabinet do?}
        {advises the President}
        \end{flashcard}\cardfrontfoot{Question 36}
        \begin{flashcard}[\tiny AMERICAN GOVERNMENT: B: System of Government]{36. What are two Cabinet-level positions?}
        {Secretary of Agriculture\footnotesize, Secretary of Commerce\footnotesize, Secretary of Defense\footnotesize, Secretary of Education\footnotesize, Secretary of Energy\footnotesize, Secretary of Health and Human Services\footnotesize, Secretary of Homeland Security\footnotesize, Secretary of Housing and Urban Development\footnotesize, Secretary of the Interior\footnotesize, Secretary of Labor\footnotesize, Secretary of State\footnotesize, Secretary of Transportation\footnotesize, Secretary of the Treasury\footnotesize, Secretary of Veterans Affairs\footnotesize, Attorney General\footnotesize, Vice President}
        \end{flashcard}\cardfrontfoot{Question 37}
        \begin{flashcard}[\tiny AMERICAN GOVERNMENT: B: System of Government]{37. What does the judicial branch do?}
        {reviews laws\\explains laws\\resolves disputes (disagreements)\\decides if a law goes against the Constitution}
        \end{flashcard}\cardfrontfoot{Question 38}
        \begin{flashcard}[\tiny AMERICAN GOVERNMENT: B: System of Government]{38. What is the highest court in the United States?}
        {the Supreme Court}
        \end{flashcard}\cardfrontfoot{Question 39}
        \begin{flashcard}[\tiny AMERICAN GOVERNMENT: B: System of Government]{39. How many justices are on the Supreme Court?}
        {nine (9){\footnotesize{\textsl{Updated on 2022-10-04}}}}
        \end{flashcard}\cardfrontfoot{Question 40}
        \begin{flashcard}[\tiny AMERICAN GOVERNMENT: B: System of Government]{40. Who is the Chief Justice of the United States now?}
        {John Roberts\\John G. Roberts, Jr.{\footnotesize{\textsl{Updated on 2022-10-04}}}}
        \end{flashcard}\cardfrontfoot{Question 41}
        \begin{flashcard}[\tiny AMERICAN GOVERNMENT: B: System of Government]{41. Under our Constitution, some powers belong to the federal government. What is one power of the federal government?}
        {to print money\\to declare war\\to create an army\\to make treaties}
        \end{flashcard}\cardfrontfoot{Question 42}
        \begin{flashcard}[\tiny AMERICAN GOVERNMENT: B: System of Government]{42. Under our Constitution, some powers belong to the states. What is one power of the states?}
        {provide schooling and education\\provide protection (police)\\provide safety (fire departments)\\give a driver’s license\\approve zoning and land use}
        \end{flashcard}\cardfrontfoot{Question 43}
        \begin{flashcard}[\tiny AMERICAN GOVERNMENT: B: System of Government]{43. Who is the Governor of your state now?}
        {Albert Bryan{\footnotesize{\textsl{Updated for VI on 2022-10-04}}}}
        \end{flashcard}\cardfrontfoot{Question 44}
        \begin{flashcard}[\tiny AMERICAN GOVERNMENT: B: System of Government]{44. What is the capital of your state?*}
        {Trick question. DC isn't a state}
        \end{flashcard}\cardfrontfoot{Question 45}
        \begin{flashcard}[\tiny AMERICAN GOVERNMENT: B: System of Government]{45. What are the two major political parties in the United States?*}
        {Democratic and Republican{\footnotesize{\textsl{Updated on 2022-10-04}}}}
        \end{flashcard}\cardfrontfoot{Question 46}
        \begin{flashcard}[\tiny AMERICAN GOVERNMENT: B: System of Government]{46. What is the political party of the President now?}
        {Democratic (Party){\footnotesize{\textsl{Updated on 2022-10-04}}}}
        \end{flashcard}\cardfrontfoot{Question 47}
        \begin{flashcard}[\tiny AMERICAN GOVERNMENT: B: System of Government]{47. What is the name of the Speaker of the House of Representatives now?}
        {Nancy Pelosi\\Pelosi}
        \end{flashcard}\cardfrontfoot{Question 48}
        \begin{flashcard}[\tiny AMERICAN GOVERNMENT: C: Rights and Responsibilities]{48. There are four amendments to the Constitution about who can vote. Describe one of them.}
        {Citizens eighteen (18) and older (can vote).\\You don’t have to pay (a poll tax) to vote.\\Any citizen can vote. (Women and men can vote.)\\A male citizen of any race (can vote).}
        \end{flashcard}\cardfrontfoot{Question 49}
        \begin{flashcard}[\tiny AMERICAN GOVERNMENT: C: Rights and Responsibilities]{49. What is one responsibility that is only for United States citizens?*}
        {serve on a jury\\vote in a federal election}
        \end{flashcard}\cardfrontfoot{Question 50}
        \begin{flashcard}[\tiny AMERICAN GOVERNMENT: C: Rights and Responsibilities]{50. Name one right only for United States citizens.}
        {vote in a federal election\\run for federal office}
        \end{flashcard}\cardfrontfoot{Question 51}
        \begin{flashcard}[\tiny AMERICAN GOVERNMENT: C: Rights and Responsibilities]{51. What are two rights of everyone living in the United States?}
        {freedom of expression\\freedom of speech\\freedom of assembly\\freedom to petition the government\\freedom of religion\\the right to bear arms}
        \end{flashcard}\cardfrontfoot{Question 52}
        \begin{flashcard}[\tiny AMERICAN GOVERNMENT: C: Rights and Responsibilities]{52. What do we show loyalty to when we say the Pledge of Allegiance?}
        {the United States\\the flag}
        \end{flashcard}\cardfrontfoot{Question 53}
        \begin{flashcard}[\tiny AMERICAN GOVERNMENT: C: Rights and Responsibilities]{53. What is one promise you make when you become a United States citizen?}
        {give up loyalty to other countries\\defend the Constitution and laws of the United States\\obey the laws of the United States\\serve in the U.S. military (if needed)\\serve (do important work for) the nation (if needed)\\be loyal to the United States}
        \end{flashcard}\cardfrontfoot{Question 54}
        \begin{flashcard}[\tiny AMERICAN GOVERNMENT: C: Rights and Responsibilities]{54. How old do citizens have to be to vote for President?*}
        {eighteen (18) and older}
        \end{flashcard}\cardfrontfoot{Question 55}
        \begin{flashcard}[\tiny AMERICAN GOVERNMENT: C: Rights and Responsibilities]{55. What are two ways that Americans can participate in their democracy?}
        {vote\footnotesize, join a political party\footnotesize, help with a campaign\footnotesize, join a civic group\footnotesize, join a community group\footnotesize, give an elected official your opinion on an issue\footnotesize, call Senators and Representatives\footnotesize, publicly support or oppose an issue or policy\footnotesize, run for office\footnotesize, write to a newspaper}
        \end{flashcard}\cardfrontfoot{Question 56}
        \begin{flashcard}[\tiny AMERICAN GOVERNMENT: C: Rights and Responsibilities]{56. When is the last day you can send in federal income tax forms?*}
        {April 15}
        \end{flashcard}\cardfrontfoot{Question 57}
        \begin{flashcard}[\tiny AMERICAN GOVERNMENT: C: Rights and Responsibilities]{57. When must all men register for the Selective Service?}
        {at age eighteen (18)\\between eighteen (18) and twenty-six (26)}
        \end{flashcard}\cardfrontfoot{Question 58}
        \begin{flashcard}[\tiny AMERICAN HISTORY: A: Colonial Period and Independence]{58. What is one reason colonists came to America?}
        {freedom\\political liberty\\religious freedom\\economic opportunity\\practice their religion\\escape persecution}
        \end{flashcard}\cardfrontfoot{Question 59}
        \begin{flashcard}[\tiny AMERICAN HISTORY: A: Colonial Period and Independence]{59. Who lived in America before the Europeans arrived?}
        {American Indians\\Native Americans}
        \end{flashcard}\cardfrontfoot{Question 60}
        \begin{flashcard}[\tiny AMERICAN HISTORY: A: Colonial Period and Independence]{60. What group of people was taken to America and sold as slaves?}
        {Africans\\people from Africa}
        \end{flashcard}\cardfrontfoot{Question 61}
        \begin{flashcard}[\tiny AMERICAN HISTORY: A: Colonial Period and Independence]{61. Why did the colonists fight the British?}
        {because of high taxes (taxation without representation)\\because the British army stayed in their houses (boarding, quartering)\\because they didn’t have self-government}
        \end{flashcard}\cardfrontfoot{Question 62}
        \begin{flashcard}[\tiny AMERICAN HISTORY: A: Colonial Period and Independence]{62. Who wrote the Declaration of Independence?}
        {(Thomas) Jefferson}
        \end{flashcard}\cardfrontfoot{Question 63}
        \begin{flashcard}[\tiny AMERICAN HISTORY: A: Colonial Period and Independence]{63. When was the Declaration of Independence adopted?}
        {July 4, 1776}
        \end{flashcard}\cardfrontfoot{Question 64}
        \begin{flashcard}[\tiny AMERICAN HISTORY: A: Colonial Period and Independence]{64. There were 13 original states. Name three.}
        {New Hampshire\footnotesize, Massachusetts\footnotesize, Rhode Island\footnotesize, Connecticut\footnotesize, New York\footnotesize, New Jersey\footnotesize, Pennsylvania\footnotesize, Delaware\footnotesize, Maryland\footnotesize, Virginia\footnotesize, North Carolina\footnotesize, South Carolina\footnotesize, Georgia}
        \end{flashcard}\cardfrontfoot{Question 65}
        \begin{flashcard}[\tiny AMERICAN HISTORY: A: Colonial Period and Independence]{65. What happened at the Constitutional Convention?}
        {The Constitution was written.\\The Founding Fathers wrote the Constitution.}
        \end{flashcard}\cardfrontfoot{Question 66}
        \begin{flashcard}[\tiny AMERICAN HISTORY: A: Colonial Period and Independence]{66. When was the Constitution written?}
        {1787}
        \end{flashcard}\cardfrontfoot{Question 67}
        \begin{flashcard}[\tiny AMERICAN HISTORY: A: Colonial Period and Independence]{67. The Federalist Papers supported the passage of the U.S. Constitution. Name one of the writers.}
        {(James) Madison\\(Alexander) Hamilton\\(John) Jay\\Publius}
        \end{flashcard}\cardfrontfoot{Question 68}
        \begin{flashcard}[\tiny AMERICAN HISTORY: A: Colonial Period and Independence]{68. What is one thing Benjamin Franklin is famous for?}
        {U.S. diplomat\\oldest member of the Constitutional Convention\\first Postmaster General of the United States\\writer of “Poor Richard’s Almanac”\\started the first free libraries}
        \end{flashcard}\cardfrontfoot{Question 69}
        \begin{flashcard}[\tiny AMERICAN HISTORY: A: Colonial Period and Independence]{69. Who is the “Father of Our Country”?}
        {(George) Washington}
        \end{flashcard}\cardfrontfoot{Question 70}
        \begin{flashcard}[\tiny AMERICAN HISTORY: A: Colonial Period and Independence]{70. Who was the first President?*}
        {(George) Washington}
        \end{flashcard}\cardfrontfoot{Question 71}
        \begin{flashcard}[\tiny AMERICAN HISTORY: B: 1800s]{71. What territory did the United States buy from France in 1803?}
        {the Louisiana Territory\\Louisiana}
        \end{flashcard}\cardfrontfoot{Question 72}
        \begin{flashcard}[\tiny AMERICAN HISTORY: B: 1800s]{72. Name one war fought by the United States in the 1800s.}
        {War of 1812\\Mexican-American War\\Civil War\\Spanish-American War}
        \end{flashcard}\cardfrontfoot{Question 73}
        \begin{flashcard}[\tiny AMERICAN HISTORY: B: 1800s]{73. Name the U.S. war between the North and the South.}
        {the Civil War\\the War between the States}
        \end{flashcard}\cardfrontfoot{Question 74}
        \begin{flashcard}[\tiny AMERICAN HISTORY: B: 1800s]{74. Name one problem that led to the Civil War.}
        {slavery\\economic reasons\\states’ rights}
        \end{flashcard}\cardfrontfoot{Question 75}
        \begin{flashcard}[\tiny AMERICAN HISTORY: B: 1800s]{75. What was one important thing that Abraham Lincoln did?*}
        {freed the slaves (Emancipation Proclamation)\\saved (or preserved) the Union\\led the United States during the Civil War}
        \end{flashcard}\cardfrontfoot{Question 76}
        \begin{flashcard}[\tiny AMERICAN HISTORY: B: 1800s]{76. What did the Emancipation Proclamation do?}
        {freed the slaves\\freed slaves in the Confederacy\\freed slaves in the Confederate states\\freed slaves in most Southern states}
        \end{flashcard}\cardfrontfoot{Question 77}
        \begin{flashcard}[\tiny AMERICAN HISTORY: B: 1800s]{77. What did Susan B. Anthony do?}
        {fought for women’s rights\\fought for civil rights}
        \end{flashcard}\cardfrontfoot{Question 78}
        \begin{flashcard}[\tiny AMERICAN HISTORY: C: Recent American History and Other Important Historical Information]{78. Name one war fought by the United States in the 1900s.*}
        {World War I\\World War II\\Korean War\\Vietnam War\\(Persian) Gulf War}
        \end{flashcard}\cardfrontfoot{Question 79}
        \begin{flashcard}[\tiny AMERICAN HISTORY: C: Recent American History and Other Important Historical Information]{79. Who was President during World War I?}
        {(Woodrow) Wilson}
        \end{flashcard}\cardfrontfoot{Question 80}
        \begin{flashcard}[\tiny AMERICAN HISTORY: C: Recent American History and Other Important Historical Information]{80. Who was President during the Great Depression and World War II?}
        {(Franklin) Roosevelt}
        \end{flashcard}\cardfrontfoot{Question 81}
        \begin{flashcard}[\tiny AMERICAN HISTORY: C: Recent American History and Other Important Historical Information]{81. Who did the United States fight in World War II?}
        {Japan, Germany, and Italy}
        \end{flashcard}\cardfrontfoot{Question 82}
        \begin{flashcard}[\tiny AMERICAN HISTORY: C: Recent American History and Other Important Historical Information]{82. Before he was President, Eisenhower was a general. What war was he in?}
        {World War II}
        \end{flashcard}\cardfrontfoot{Question 83}
        \begin{flashcard}[\tiny AMERICAN HISTORY: C: Recent American History and Other Important Historical Information]{83. During the Cold War, what was the main concern of the United States?}
        {Communism}
        \end{flashcard}\cardfrontfoot{Question 84}
        \begin{flashcard}[\tiny AMERICAN HISTORY: C: Recent American History and Other Important Historical Information]{84. What movement tried to end racial discrimination?}
        {civil rights (movement)}
        \end{flashcard}\cardfrontfoot{Question 85}
        \begin{flashcard}[\tiny AMERICAN HISTORY: C: Recent American History and Other Important Historical Information]{85. What did Martin Luther King, Jr. do?*}
        {fought for civil rights\\worked for equality for all Americans}
        \end{flashcard}\cardfrontfoot{Question 86}
        \begin{flashcard}[\tiny AMERICAN HISTORY: C: Recent American History and Other Important Historical Information]{86. What major event happened on September 11, 2001, in the United States?}
        {Terrorists attacked the United States.}
        \end{flashcard}\cardfrontfoot{Question 87}
        \begin{flashcard}[\tiny AMERICAN HISTORY: C: Recent American History and Other Important Historical Information]{87. Name one American Indian tribe in the United States.}
        {\[USCIS Officers will be supplied with a list of federally recognized American Indian tribes.\]\footnotesize, Cherokee\footnotesize, Navajo\footnotesize, Sioux\footnotesize, Chippewa\footnotesize, Choctaw\footnotesize, Pueblo\footnotesize, Apache\footnotesize, Iroquois\footnotesize, Creek\footnotesize, Blackfeet\footnotesize, Seminole\footnotesize, Cheyenne\footnotesize, Arawak\footnotesize, Shawnee\footnotesize, Mohegan\footnotesize, Huron\footnotesize, Oneida\footnotesize, Lakota\footnotesize, Crow...and 3 more}
        \end{flashcard}\cardfrontfoot{Question 88}
        \begin{flashcard}[\tiny INTEGRATED CIVICS: A: Geography]{88. Name one of the two longest rivers in the United States.}
        {Missouri (River)\\Mississippi (River)}
        \end{flashcard}\cardfrontfoot{Question 89}
        \begin{flashcard}[\tiny INTEGRATED CIVICS: A: Geography]{89. What ocean is on the West Coast of the United States?}
        {Pacific (Ocean)}
        \end{flashcard}\cardfrontfoot{Question 90}
        \begin{flashcard}[\tiny INTEGRATED CIVICS: A: Geography]{90. What ocean is on the East Coast of the United States?}
        {Atlantic (Ocean)}
        \end{flashcard}\cardfrontfoot{Question 91}
        \begin{flashcard}[\tiny INTEGRATED CIVICS: A: Geography]{91. Name one U.S. territory.}
        {Puerto Rico\\U.S. Virgin Islands\\American Samoa\\Northern Mariana Islands\\Guam}
        \end{flashcard}\cardfrontfoot{Question 92}
        \begin{flashcard}[\tiny INTEGRATED CIVICS: A: Geography]{92. Name one state that borders Canada.}
        {Maine\footnotesize, New Hampshire\footnotesize, Vermont\footnotesize, New York\footnotesize, Pennsylvania\footnotesize, Ohio\footnotesize, Michigan\footnotesize, Minnesota\footnotesize, North Dakota\footnotesize, Montana\footnotesize, Idaho\footnotesize, Washington\footnotesize, Alaska}
        \end{flashcard}\cardfrontfoot{Question 93}
        \begin{flashcard}[\tiny INTEGRATED CIVICS: A: Geography]{93. Name one state that borders Mexico.}
        {California\\Arizona\\New Mexico\\Texas}
        \end{flashcard}\cardfrontfoot{Question 94}
        \begin{flashcard}[\tiny INTEGRATED CIVICS: A: Geography]{94. What is the capital of the United States?*}
        {Washington, D.C.}
        \end{flashcard}\cardfrontfoot{Question 95}
        \begin{flashcard}[\tiny INTEGRATED CIVICS: A: Geography]{95. Where is the Statue of Liberty?*}
        {New York (Harbor)\\Liberty Island\\\[Also acceptable are New Jersey, near New York City, and on the Hudson (River).\]}
        \end{flashcard}\cardfrontfoot{Question 96}
        \begin{flashcard}[\tiny INTEGRATED CIVICS: B: Symbols]{96. Why does the flag have 13 stripes?}
        {because there were 13 original colonies\\because the stripes represent the original colonies}
        \end{flashcard}\cardfrontfoot{Question 97}
        \begin{flashcard}[\tiny INTEGRATED CIVICS: B: Symbols]{97. Why does the flag have 50 stars?*}
        {because there is one star for each state\\because each star represents a state\\because there are 50 states}
        \end{flashcard}\cardfrontfoot{Question 98}
        \begin{flashcard}[\tiny INTEGRATED CIVICS: B: Symbols]{98. What is the name of the national anthem?}
        {The Star-Spangled Banner}
        \end{flashcard}\cardfrontfoot{Question 99}
        \begin{flashcard}[\tiny INTEGRATED CIVICS: C: Holidays]{99. When do we celebrate Independence Day?*}
        {July 4}
        \end{flashcard}\end{document}